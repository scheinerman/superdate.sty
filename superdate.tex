\documentclass[12pt]{amsart}
\usepackage{superdate}
\usepackage[margin=1.25in]{geometry}
\usepackage{txfonts}

\title{The \texttt{superdate} Package}

\author{Edward R. Scheinerman}
\thanks{%
    Department of Applied Mathematics
    and Statistics, Johns Hopkins University, Baltimore, Maryland 21218
    USA.}


\date{\superdate}

\begin{document}

\maketitle

\section{Purpose}

The \verb|superdate| package provides the \verb|\superdate| macro that
produces the current date and time in the format
\verb|YEAR:MM:DD:hh:mm| where
\begin{itemize}
\item \verb|YEAR| is the current year,
\item \verb|MM| is the current month (from 01 to 12),
\item \verb|DD| is the current day (from 01 to 31),
\item \verb|hh| is the current hour (from 00 to 23), and 
\item \verb|mm| is the current minute (from 00 to 59).
\end{itemize}

This is a sensible format for date/time that puts the most significant
data to the left and the least significant data to the right. 

It is useful to date papers with \verb|\superdate|, especially when
producing multiple versions on a given day.

The separating character is, by default, a colon. This can be changed
by redefining the macro \verb|\superdateseparator| like this:
\begin{verbatim}
\renewcommand{\superdateseparator}{/}
\end{verbatim}
\renewcommand{\superdateseparator}{/}%
Now the output of \verb|\superdate| looks like
this: \superdate.


\section{Other user-level macros}
\label{sect:user}


A few other handy date/time macros are provided.
\begin{itemize}
\item The macro \verb|\hyphendate| produces the current date in the
  format \verb|YEAR-MM-DD| with leading zeros inserted into the month
  and date if they are less than 10. Example: Today is \hyphendate.


\item The macro \verb|\ampmtime| produces the current time in 12-hour
  clock format with the letters `am' or `pm' appended. For example,
  the current time is \ampmtime. Note that the hour may appear as a
  single digit, but the minute will always be expressed with two
  digits by prepending a zero as needed.

\item The macro \verb|\miltime| produces the current time in 24-hour
  format. Example: the current time is \miltime. Both the hour and the
  minute will be two digits.
\end{itemize}


\section{Setting a different date and time}

The date and time used by \verb|\superdate| (and the other macros
described in \S\ref{sect:user}) are set from the system clock at the
time the \verb|superdate| package is loaded; that is, at the beginning
of the \LaTeX\ run.

The macro \verb|\setcl@ck| is provided so users can reset the clock
later in the document, or to set a date/time different from the
current date/time. The syntax is:

\noindent%
  \verb|\setcl@ck{|\emph{time}\verb|}{|\emph{month}%
    \verb|}{|\emph{day}\verb|}{|\emph{year}\verb|}|

\noindent where 
\begin{itemize}
\item \emph{time} is the  time expressed as the number of
  minutes past midnight,
\item \emph{month} is the month expressed as number between 1 and 12,
\item \emph{day} is the day of the month expressed as a number between
  1 and 31, and
\item \emph{year} is the year.
\end{itemize}
This command should be enclosed between \verb|\makeatletter| and
\verb|\makeatother|. For example
\begin{verbatim}
\makeatletter
\setcl@ck{721}{5}{24}{2009}
\makeatother
\superdate
\end{verbatim}
% Produces: 
\makeatletter%
\setcl@ck{721}{5}{24}{2009}%
\makeatother%
\renewcommand\superdateseparator{:}%
produces \superdate.

Note that this does not modify the output of \verb|\today| which is \today.

\end{document}
